%%%%%%%%%%%%%%%%%%%%%%%%%%%%%%%%%%%%%%%%%
% Twenty Seconds Resume/CV
% LaTeX Template
% Version 1.0 (14/7/16)
%
% Original author:
% Carmine Spagnuolo (cspagnuolo@unisa.it) with major modifications by 
% Vel (vel@LaTeXTemplates.com) and Harsh (harsh.gadgil@gmail.com)
%
% License:
% The MIT License (see included LICENSE file)
%
%%%%%%%%%%%%%%%%%%%%%%%%%%%%%%%%%%%%%%%%%
% https://github.com/naveenkumardb/Machine-Learning-Engineer-Resume-LaTeX
%----------------------------------------------------------------------------------------
%	PACKAGES AND OTHER DOCUMENT CONFIGURATIONS
%----------------------------------------------------------------------------------------

\documentclass[letterpaper]{twentysecondcv} % a4paper for A4
\usepackage{array}
\usepackage{ragged2e}
\usepackage{xcolor}


% Command for printing skill overview bubbles
\newcommand\skills{
\begin{skillslist}
  \item AI
  \begin{itemize}
    \item Machine Learning
    \begin{itemize}
      \item Deep Learning
      \item Reinforcement Learning
    \end{itemize}
  \end{itemize}
    \item Statistical Data Analysis
   \item Algorithms
   \begin{itemize}
    \item Mathematical Analysis
    \item Parallel | Distributed
   \end{itemize}
\end{skillslist}
% ~
% 	\smartdiagram[bubble diagram]{
%         \textbf{Data}\\\textbf{Science},
%         \textbf{~Machine~}\\\textbf{Learning},
%         \textbf{~Re~}\\\textbf{Visualization},
%         \textbf{~~Machine~~}\\\textbf{Learning},
%         \textbf{~Data~}\\\textbf{~Wrangling~}
%     }
}


% Projects text
\education{
\vspace{.1em}\\
\centerline{\textbf{Master of Science Degrees:}}\\
\centerline{\textbf{GPA 3.87}}\\
\centerline{\textbf{Computer Science | Mathematics}}\\
Western Washington University\\
Bellingham, WA

\textbf{Bachelor of Science} | \textbf{Bachelor of Arts:}\\
\centerline{\textbf{GPA 4.0}}\\
\centerline{\textbf{Physics} | \textbf{Computer Science}}\\
The Evergreen State College\\
Olympia, WA\\
}
\career{
\textbf{Western Washington University} \\
Graduate Research Assistant \\
\textbf{Duration:} Mar. 2022 - Present\\
Graduate Teaching Assistant \\
\textbf{Duration:} Oct. 2021 - Mar. 2022 \vspace{2mm}\\
\textbf{The Evergreen State College}\\
Teaching Assistant\\
\textbf{Duration:} Oct. 2020 - Jun. 2021 \vspace{2mm}\\
\textbf{College Tutor}:\\
Bellingham Technical College:\\
\textbf{Duration:} Mar. 2020 - Mar. 2022\\
Skagit Vallege College:\\
\textbf{Duration:} Dec. 2019 - Jun. 2021\\
}


%----------------------------------------------------------------------------------------
%	 PERSONAL INFORMATION
%----------------------------------------------------------------------------------------
% If you don't need one or more of the below, just remove the content leaving the command, e.g. \cvnumberphone{}

\cvname{\huge Seth Briney} % Your name
\cvjobtitle{\large Machine Learning Engineer} % Job
% title/career

\cvlinkedin{/in/SethLBriney}
\cvgithub{Seth1Briney}
\cvnumberphone{+1 (360)708-1539} % Phone number
\cvsite{SethBriney.com} % Personal website
\cvmail{SethLBriney@gmail.com} % Email address

%----------------------------------------------------------------------------------------
\definecolor{ForestGreen}{HTML}{33912B}

\begin{document}

\makeprofile % Print the sidebar

%----------------------------------------------------------------------------------------
%	 SUMMARY
%----------------------------------------------------------------------------------------

Passionate about scientific computing and AI, I am goal oriented and have a strong drive to expand, deepen, and apply my diverse skill-set toward delivering impactful results and innovative solutions to real-world problems. I possess both a strong theoretical background, and a proven aptitude for developing computational programs and algorithms to achieve high quality performance metrics in machine learning problems.
\section{L\textcolor{ForestGreen}{ang}uages, \textcolor{ForestGreen}{Mod}ules, and \textcolor{ForestGreen}{Alg}orithms}
*skill* means strong familiarity, **skill** means mastered.
\begin{itemize}
\item Python modules including: \textbf{Gymnasium/OpenAiGym, MatPlotLib, **Numpy**, *Pandas*, PyGame, SciKit-Learn, *SKRL*, TensorFlow, *WanDB*, **PyTorch**}
\item Computational languages including: \textbf{Julia, *MATLAB/Octave*, R}
\item General languages including: \textbf{*BASH*, *C*, C++, C\#, Java, **Python**}
\item Algorithms including: \textbf{Gaussian Mixture Model clustering, Expectation Maximization, Inverse Transform Sampling, Matrix Analytic Methods using SVD, QR factorization, Rejection Sampling, Red/Black SOR, Stochastic Integration}
\item Mathematics and science disciplines including: \textbf{**Analysis**, *Algebra*, /*Calculus*, Chemistry, *Physics*, *Statistics*}
\item Computational techniques such as: \textbf{Distributed Computing, HPC, Meta Learning, Parallel Computing, Transfer Learning}
\item Miscelaneous \textbf{*Bayesian Decision Theory*, *ChatGPT* Docker, EnergyPlus, Excel, Git, *LaTex*, Project Collaboration, *SQL*}
\end{itemize}
\begin{itemize}
\item I am very interested in learning more about: bci, biology, biogerontology, ecology, investing, music theory, and neuroscience. 
\end{itemize}
\section{E\textcolor{ForestGreen}{xp}erience}
\setlength{\tabcolsep}{8pt}  % increase the left and right padding
\renewcommand{\arraystretch}{1.5} % increase the vertical padding
\begin{tabular}{|>{\centering\arraybackslash}m{0.2\textwidth}|>{\RaggedRight\arraybackslash}m{0.75\textwidth}|}
\hline
Time-frame & Summary\\\hline
1 year & Worked in a funded research collaboration with \textbf{PNNL} focused on electric load forecasting for commercial office buildings using \textbf{EnergyPlus} for simulation and \textbf{PyTorch} for \textbf{deep learning}. Utilized \textbf{HTCondor} for \textbf{distributed} \textbf{HPC}, and \textbf{Weights and Biases} \textbf{for hyper-parameter} tuning in \textbf{meta/transfer-learning} experiments.\\\hline
9 months & In another funded collaboration with \textbf{PNNL}, applied \textbf{deep reinforcement learning} toward achieving \textbf{optimal control} in a variety of \textbf{physics} simulators including \textbf{building energy} simulations. Contributed to the \textbf{Neuromancer} open-source project, acting as the lead programmer in \textbf{PslGym} to wrap \textbf{Neuromancer PSL Nonautonomous} systems in a \textbf{Gymnasium} interface for deep reinforcement learning control implemented with the \textbf{SKRL Python} module. Created \textbf{graphical} frameworks for debugging and \textbf{visualizing results} including an interactive TwoTank control game, using \textbf{Matplotlib} and \textbf{Pygame}. Achieved control with algorithms including: \textbf{A2C}, \textbf{DDPG}, \textbf{PPO}, \textbf{TD3}, and \textbf{TRPO}; in reference tracking and energy management physics control problems. Added random walk variants to the \textbf{Signals} nm sub-module for \textbf{robust sampling} in \textbf{data generation}.\\\hline
1 year & Applied a variety of \textbf{tabular reinforcement learning} techniques including \textbf{Q-learning} and \textbf{Dyna-Q}, to a variety of academic \textbf{control problems} including \textbf{windy grid-world} and \textbf{maze-running}, following \textbf{Sutton/Barto}'s Introduction to Reinforcement Learning.\\\hline
2 years & \textbf{Collaborated} with a team of \textbf{college tutors}, helping students at \textbf{BTC} and \textbf{SCV} in various technical subjects especially \textbf{algebra}, \textbf{calculus}, \textbf{chemistry}, \textbf{computer science}, and \textbf{physics}. Team members would often pass students to me with the more \textbf{challenging math} problems.\\\hline
%3 months & Created an interactive ballistic physics simulator game with reinforcement learning control where two balls play tag, up to one controlled by the user at a time.\\\hline
\end{tabular}
\end{document} 
